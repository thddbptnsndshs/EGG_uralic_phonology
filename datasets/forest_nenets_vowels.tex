% !TEX TS-program = XeLaTeX

% STYLE

\documentclass[a4paper, 12pt]{article}
\usepackage[left=1in,
		    right=1in,
    		    top=1in,
		    bottom=1in,
		    bindingoffset=0cm]{geometry}
		    \usepackage{array}
\usepackage{float}
\usepackage{graphicx}
\graphicspath{ {./images/} }
\usepackage{subfig}
\usepackage{enumerate}
\usepackage[normalem]{ulem} % underlining
\usepackage{booktabs} % tables
\usepackage[table]{xcolor} % coloring tables
\newcolumntype{L}[1]{>{\raggedright\let\newline\\\arraybackslash\hspace{0pt}}m{#1}} % beautiful column types
\newcolumntype{C}[1]{>{\centering\let\newline\\\arraybackslash\hspace{0pt}}m{#1}}

% LANGUAGE + FONT
		    
\usepackage[english]{babel}
\usepackage[backend=biber,
                     style=unified]{biblatex}
\newcommand{\citeay}[2][]{
   \citeauthor{#2} (\citeyear[#1]{#2})}
\addbibresource{ref.bib}
\usepackage{fontspec}  
\usepackage{hyperref}
\hypersetup{
    colorlinks=true,
    linkcolor=black,
    citecolor=black,
    filecolor=black,
    urlcolor=blue,
} 
\setmainfont{Minion 3}

% DRAWING

\usepackage{tikz}
\usepackage{tikz-qtree}
\usetikzlibrary{shapes.geometric}
\usetikzlibrary{trees,arrows}
\usetikzlibrary{positioning}
\usetikzlibrary{matrix}
\usetikzlibrary{tikzmark}
\usetikzlibrary{decorations.shapes}

% LINGUISTICS 

\usepackage{expex}
\usepackage[glossaries]{leipzig}
\makeglossaries
\newleipzig {npst} {npst} {non-past}
\newleipzig {nfin} {nfin} {non-finite}
\newleipzig {nsg} {nsg} {non-singular}
\newleipzig {prol} {prol} {prolative case}
\newleipzig {el} {el} {elative case}

\lingset{numoffset=1ex, aboveexskip=1em, belowexskip=1em}

\title{Forest Nenets dataset}
\author{EGG2023, Novi Sad}
\date{Last updated \today}

\begin{document}
\maketitle

	\section{Basic facts}
	
	Consonant and vowel inventories \parencite{egorov2023}

\begin{minipage}[t]{0.45\linewidth}
\begin{table}[H]
\begin{tabular}{llll}
\toprule
i & ĭ  & u & ŭ \\
e & ĕ  & o & ŏ \\
æ & æ̆ & a & ă \\
  & ̊  &   &  \\
\bottomrule
\end{tabular}
\end{table}
\end{minipage}
\hfill
\begin{minipage}[t]{0.45\linewidth}
\begin{table}[H]
\begin{tabular}{lllllll}
\toprule
p & pˊ & t & ć              & k            & kˊ & ʔ \\
m & mˊ & n & ń              & ŋ            & ŋˊ &   \\
  &    & s & ś              & x            & xˊ &   \\
  &    & λ & λˊ             &              &    &   \\
w & wˊ & l & lˊ             &              &    &   \\
  &    &   & \multicolumn{2}{l}{dˊ$\sim$j} &    &  \\
\bottomrule
\end{tabular}
\end{table}
\end{minipage}
\vspace*{1em}

	\noindent / ̊/ = schwa, /λ/ = voiceless lateral fricative ([ɬ]), acute accents denote palatalisation, short vowels are marked with a breve (long /a/ -- short /ă/).

	\section{Data}
	
	For more detailed descriptions of the phonology of Forest Nenets, see \citeay{burkova2022} or \citeay{salminen2007} (both are on \href{https://github.com/thddbptnsndshs/EGG_uralic_phonology/blob/main/readings.md}{the reading list}). This dataset uses a slightly modified version of Tapani Salminen's practical transcription; the difference is that I mark palatalisation with acute accents.
	
		\subsection{Vowel quantity}
		
	Vowels can be long (i e æ u o a) and short (ĭ ĕ æ̆ ŭ ŏ ă  ̊ ). 
	
	\pex
		\a wĭŋ `tundra'
		\a wiŋkăt° `tundra.{\El}.{\Sg}' 
		\a ńit° `{\Neg}.{\Fsg}'		\trailingcitation{\parencite{salminen2007}}
	\xe
	
%	\pex
%		\a 
%		\a
%	\xe
		
	\begin{enumerate}[$\gg$]
		\item What positions can/cannot long vowels occur in?
		\item \citeay{salminen2007} considers long vowels unmarked and the short vowels marked, which is ``understandably a conventional and perhaps controversial decision''. Why is this interpretation still accepted despite being controversial?
	\end{enumerate}

		\subsection{Stress}
		
	Unstressed long vowels are shortened automatically, so this shortening is not reflected in the transcription.
		
%	trochee + pre-schwa syllable stressed
	\pex Basic illustration
		\a ˈćońa `fox'
		\a ˈćońaŋ `fox.{\Gen}.{\Sg}'
		\a ćoˈńan ̊ `fox.{\Dat}.{\Sg}'
		\a ˈćońaˌmăna `fox.{\Prol}.{\Sg}' \trailingcitation{\parencite{salminen2007}}
	\xe
		
%	stressed o/unstressed u
	\pex o/u alternation
		\a ˈŋănu `boat'
		\a ŋăˈnoj ̊ `boat.{\Poss}.{\Fsg}'
		\a ˈŋănuta `boat.{\Poss}.{\Tsg}' \trailingcitation{\parencite{salminen2007}}
	\xe
	
%	initial, trochee, trochee
	\pex Guess the stress \label{ex:guessstress}
		\a wejaʔku `dog'
		\a wejaʔkoj° `dog.{\Poss}.{\Fsg}'
		\a wejaʔkota `dog.{\Poss}.{\Tsg}' \trailingcitation{\parencite{salminen2007}}
	\xe
		
	\begin{enumerate}[$\gg$]
		\item Formulate the stress rule in Forest Nenets based on the provided data. Sounds familiar?
		\item What causes the o/u alternation?
		\item Where does stress fall in the three forms in example (\ref{ex:guessstress})?
	\end{enumerate}

		\subsection{Schwa}
		
	``Schwa'' is only rarely phonetically realised as an over-short vowel [ə̆] and can only be found in unstressed syllables \parencite{burkova2022}. 
	
	\pex Surface realisation of schwa
		\a tămna [tamna] `more'
		\a tem ̊na [teːmːna] `deer.{\Prol}.{\Sg}'
		\a pom ̊na [poːmːna] `among' \trailingcitation{\parencite{salminen2007}}
	\xe
		
	\begin{enumerate}[$\gg$]
		\item 
	\end{enumerate}

%		\subsection{}
%		
%	\begin{enumerate}[$\gg$]
%		\item 
%	\end{enumerate}

\printbibliography

\end{document}